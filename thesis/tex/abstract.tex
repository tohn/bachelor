\cleardoublepage\null

\vspace*{\fill}

\section*{\centering Abstract}

Bei vegan lebenden Menschen ist es gerade in der Anfangszeit, nachdem die Entscheidung getroffen
wurde von nun an vegan zu leben, schwierig, sich im Supermarkt 
zurechtzufinden, da ausgiebig die Inhaltsstoffe aller zu konsumierenden 
Produkte genaustens studiert werden müssen. Und auch nach diesem Studium kann 
es immer noch sein, dass das Produkt nicht vegan ist, da etwa tierische Stoffe 
in der Herstellung benutzt wurden, die allerdings nicht deklariert werden 
müssen.
Um dies zu klären sind üblicherweise Produktanfragen an den*die Hersteller*in 
von Nöten oder eine vorherige Recherche im Internet, sofern nicht auf dem 
Produkt explizit angegeben wurde, dass es vegan ist oder es durch gewisse 
Zutaten unvegan ist.\\
Auch die verwandten Arbeiten, die in dieser Arbeit betrachtet werden, lösen 
diese Probleme nicht, da sie nicht oder nur teilweise über vegane Daten verfügen und wenn, 
diese nicht mit Produktanfragen oder ähnlichen Quellen belegt sind. Diese Daten 
sind auch nicht lizenzfrei, d.\,h. zur freien Benutzung verfügbar.\\
Daher wird in dieser Arbeit eine lizenzfreie Produktdatenbank mit 
dem Namen "`YAVA: Yet Another Vegan App"' vorgestellt, die 
automatisch nach der Eintragung eines Produktes die Veganität, d.\,h. ob und 
wie ein Produkt vegan ist, aus den Zutateninformationen berechnen bzw. 
die Veganität durch Produktanfragen und Kommentaren belegen kann. Die 
Produktanfragen können dazu dynamisch generiert und an die 
Hersteller*innen versendet werden.\\
Durch die mobile App, die im Supermarkt dazu verwendet werden kann, ein Produkt 
anhand des Barcodes auf die Veganität hin zu überprüfen, und der Datenbank im 
Hintergrund wird das Problem gelöst, welches am Anfang beschrieben wurde.\\
Im letzten Teil dieser Arbeit wird beschrieben, wie auf der Basis dieser 
Arbeit andere Arbeiten aufgebaut werden können bzw. wie \name erweitert werden 
kann.

\vspace*{\fill}
