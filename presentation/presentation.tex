\documentclass{beamer}

%% DOCUMENTATION
% http://www2.informatik.hu-berlin.de/~mischulz/beamer.html
% ftp://ftp.mpi-sb.mpg.de/pub/tex/mirror/ftp.dante.de/pub/tex/macros/latex/contrib/beamer/doc/beameruserguide.pdf
% ftp://dante.ctan.org/tex-archive/help/Catalogue/entries/prosper.html
% http://en.wikibooks.org/wiki/LaTeX/Presentations

%% HINTS
% http://www.disk0s1.de/posts/latex/beamer-und-fill/

%% PACKAGES
\usepackage[ngerman]{babel}
\usepackage[utf8x]{inputenc}
\usepackage{amsmath,amsfonts,amssymb,multirow,hyperref}
% http://en.wikibooks.org/wiki/LaTeX/Tables
\usepackage{array} % table

%% STYLE
% http://deic.uab.es/~iblanes/beamer_gallery/index.html
\usetheme{Frankfurt}
\usecolortheme{rose}
\beamertemplatenavigationsymbolsempty
\setbeamertemplate{footline}[frame number]
\setbeamercovered{transparent}
\useoutertheme[subsection=false]{smoothbars}
\useinnertheme{rectangles}
% colors
\definecolor{up}{HTML}{E5CF83}
\definecolor{middle}{HTML}{67B821}
\definecolor{low}{HTML}{B3DB10}
\setbeamercolor{titlelike}{fg=black, bg=low}
\setbeamercolor{frametitle}{fg=black, bg=low}
\setbeamercolor{subsection in head/foot}{fg=black, bg=middle}

%% TITLEPAGE
\title[YAVA]{YAVA: Yet Another Vegan App}
\author[Y. Haupenthal]{Yannic Haupenthal}
\institute[UdS]{Universität des Saarlandes}
\date[23.05.2013]{23. Mai 2013}
\subject{YAVA}
\keywords{vegan, app, yava}

%% FIXES
% Footnote: numbers instead of letters
% http://tex.stackexchange.com/questions/68887/how-do-i-change-footnote-in-beamer
\renewcommand\thempfootnote{\arabic{mpfootnote}}

%% PICTURES
% Leider sind fast alle Bilder auskommentiert, da ich nicht die Rechte
% an diesen besitze.

%% ACTUAL PRESENTATION
\begin{document}

% Title
\frame{
	\titlepage
}

% Fundamentals
\begin{frame}{Grundlagen}
		\begin{block}{(Ovo-Lacto-)Vegetarismus}
		- Ernährungsweise\\
		- schließt Verzehr von Lebewesen aus\\
		- tierische ``Produkte'' nicht unbedingt (z.\,B. Milch, Ei,
		Honig)
	\end{block}

	\begin{block}{Veganismus}
		- Lebenseinstellung\\
		- schließt Verzehr aller tierischen ``Produkte'' aus\\
		- lehnt Nutzung von Tieren ab
	\end{block}

	\begin{block}{Veganität}
		bezeichnet wie bzw. ob etwas vegan ist
	\end{block}
\end{frame}

% Overview
\frame{
	\frametitle{Übersicht}
	\tableofcontents
}

% Motivation
\section{Motivation}
\subsection*{Inhaltsstoffe}
\begin{frame}{Inhaltsstoffe}
	\vskip0pt plus 1filll
	\begin{center}
		% Quelle: siehe unten
%		\includegraphics<1>[scale=0.15]{pics/ingredients9.jpg}
		% Quelle: siehe unten + Modifizierungen mit GIMP
%		\includegraphics<2>[scale=0.15]{pics/ingredients9-2.jpg}
	\end{center}

	\begin{itemize}
		\item Namensunklarheit, z.\,B. E-Nummern, chemische
				Bezeichnungen, CI
		\item Fremdsprachen, z.\,B. Englisch, Latein, Französisch
	\end{itemize}
	\vskip0pt plus 1filll
	\par\hrulefill\par
	\tiny{Quelle: \url{http://beautyjagd.de/2011/02/23/m-a-c-lidschatten-unflappable-aus-der-peacocky-le/}}
\end{frame}

\begin{frame}{Veganität?}
	\vskip0pt plus 1filll

	\begin{columns}
		\column{.6\textwidth}
			\begin{center}
			Aroma,\\
			Lecithin,
			Milchsäure,\\
			Mono- und Diglyceride von
			Speisefettsäuren, ...
	\end{center}
		\column{.4\textwidth}
			\begin{center}
				% Quelle: siehe unten
%				\includegraphics[scale=0.25]{pics/P_ChipsFunnyFrisch_Zutaten.jpg}
			\end{center}
	\end{columns}

	\begin{block}{Aroma}
		Ausgangsmaterialien für die Herstellung können pflanzlichen,
		tierischen oder mikrobiologischen (z. B. Hefen) Ursprungs
		sein\,\footnote{\,\url{http://de.wikipedia.org/wiki/Aroma\#Aromastoffe}}
	\end{block}

	\vskip0pt plus 1filll
	\par\hrulefill\par
	\tiny{Quelle:
	\url{http://www.lebensmittelklarheit.de/cps/rde/xchg/lebensmittelklarheit/hs.xsl/2530.htm}}
\end{frame}

\subsection*{Herstellung}
\begin{frame}{Herstellungsprozess}
	\vskip0pt plus 1filll
	\begin{columns}[c]
		\column{0.4\textwidth}
			\centering
			% Quelle: siehe unten
%			\includegraphics[scale=0.1]{pics/5010.10.jpg}
		\column{0.6\textwidth}
			Schönung bzw. Klärung von Getränken
	\end{columns}
	\begin{columns}[c]
		\column{0.6\textwidth}
			Gelatine als Trägerstoff für Vitamine
		\column{0.4\textwidth}
			\centering
			% Quelle: siehe unten
%			\includegraphics[scale=0.2]{pics/716740_1_1_detail.jpg}
	\end{columns}
	\vskip0pt plus 1filll
	\par\hrulefill\par
	\tiny{Quellen:
			\url{http://www.delinat.com/weine/5010.12.html}, 
	\url{http://www.real-drive.de/isernhagen/product/716740_1_1/Hohes+C+Multivitaminsaft}}
\end{frame}

\subsection*{Unfreiheit}
\begin{frame}{Daten}
	\vskip0pt plus 1filll
	\begin{columns}[c]
		\column{0.6\textwidth}
			\begin{itemize}
				\item GS1 verwaltet und vergibt GTINs im Einzelhandel
				\item Herausgabe der Herstellerinformationen erfolgt nur durch eingeschränkte API
			\end{itemize}
		\column{0.4\textwidth}
			\centering
			% Quelle: siehe unten (500px-Version)
%			\includegraphics[scale=0.15]{pics/{500px-Logo_GS1.svg}.png}
	\end{columns}

	\begin{alertblock}{}
		Aktuell gibt es keine freien Zutatenlisten\\
		$\rightarrow$ Closed Data
	\end{alertblock}

	\vskip0pt plus 1filll
	\par\hrulefill\par
	\tiny{Quelle:
	\url{http://de.wikipedia.org/wiki/Datei:Logo_GS1.svg}}
\end{frame}

\section{Verwandte Arbeiten}
\subsection*{related-work}
\begin{frame}{Barcoo}
	\vskip0pt plus 1filll
	\begin{columns}
		\column{.60\textwidth}
			{\large Größter Produkt-Guide in Europa}
		\column{.40\textwidth}
			\centering
			% Quelle: siehe unten (500px-Version)
%			\includegraphics[scale=0.2]{pics/{500px-Barcoo_logo.svg}.png}
	\end{columns}

	\begin{exampleblock}{}
		+ benutzt vegane Daten von ``Rezeptefuchs''\\
		+ plattformunabhängig
	\end{exampleblock}

	\begin{alertblock}{}
		- keine Zutaten\\
		- kein Userinput\\
		- unfrei
	\end{alertblock}

	\vskip0pt plus 1filll
	\par\hrulefill\par
	\tiny{Quelle:
	\url{https://de.wikipedia.org/wiki/Datei:Barcoo_logo.svg}}
\end{frame}

\begin{frame}{das-ist-drin}
	\vskip0pt plus 1filll
	\begin{columns}
		\column{.60\textwidth}
			{\large Online-Verbraucherportal rund um Inhalts- und Zusatzstoffe
			von Lebensmitteln}
		\column{.40\textwidth}
			\centering
			% Quelle: siehe unten (300dpi, CMYK, JPEG)
%			\includegraphics{pics/{das-ist-drin.de_logo_cmyk}.jpg}
	\end{columns}
	
	\begin{exampleblock}{}
		+ enthält Daten zur Veganität\\
		+ Zutatenauflistung
	\end{exampleblock}

	\begin{alertblock}{}
		- Veganität nur auf der Website\\
		- nicht plattformunabhängig\\
		- unfrei\\
	\end{alertblock}
	\vskip0pt plus 1filll
	\par\hrulefill\par
	\tiny{Quelle:
	\url{http://das-ist-drin.de/ueber-uns/presse/pressematerial/}}
\end{frame}

\begin{frame}{EuLa-Armband}
	\vskip0pt plus 1filll
	\begin{columns}
		\column{.60\textwidth}
			{\large Einkaufunterstützendes Lebensmittelallergiker-Armband}
		\column{.40\textwidth}
			\centering
			% Quelle: siehe unten
%			\includegraphics[scale=0.1]{pics/eula-prototyp.png}
	\end{columns}

	\begin{exampleblock}{}
		+ RFID statt Barcode\\
		+ Fokus auf Allergien
	\end{exampleblock}
	
	\begin{alertblock}{}
		- Daten nicht veränderbar\\
		- keine veganen Angaben\\
		- unfrei\\
		- nicht plattformunabhängig
	\end{alertblock}

	\vskip0pt plus 1filll
	\par\hrulefill\par
	\tiny{Quellen:
	\url{http://kola.opus.hbz-nrw.de/volltexte/2007/180/pdf/Diplomarbeit-RFID_Lebensmittelallergie.pdf}}\\
	Bretz, A. (2007). \textbf{RFID als Technik für Mobile Health bei
	Lebensmittelallergikern.}
\end{frame}

\begin{frame}{WikiFood}
	\vskip0pt plus 1filll
	\begin{columns}
		\column{.60\textwidth}
		{\large Website innerhalb des Projekts MENSSANA (Mobile Expert and Networking System for Systematical
		Analysis of Nutrition-based Allergies)}
		\column{.40\textwidth}
			\centering
			% Quelle: siehe unten
%			\includegraphics[scale=0.5]{pics/23Logo.png}
	\end{columns}

	\begin{exampleblock}{}
		+ Fokus auf Allergien\\
		+ Veganangaben in Arbeit
	\end{exampleblock}
	
	\begin{alertblock}{}
		- unfrei\\
		- nicht plattformunabhängig
	\end{alertblock}

	\vskip0pt plus 1filll
	\par\hrulefill\par
	\tiny{Quellen:
	\url{http://www.wikifood.eu/wikifood/en/struts/welcome.do}}\\
	Arens, A., Rösch, N., Feidert, F., Harpes, P., Herbst, R., \&
	Mösges, R. \textbf{Mobile electronic patient diaries with barcode based
	food identification for the treatment of food allergies.} \textit{GMS Med.
	Inform. Biom. Epidemiol, 4(3).}
\end{frame}

\section{Ziel}
\subsection*{Ziel}
\begin{frame}{Ziel}
	\vskip0pt plus 1filll
	\begin{exampleblock}{}
		Freie, plattformübergreifende, vegane Produktdatenbank und
		Informationsplattform
	\end{exampleblock}
	\vskip0pt plus 1filll
\end{frame}

\begin{frame}{Veganismus}
	\vskip0pt plus 1filll
	\begin{columns}
		\column{.60\textwidth}
			\centering
			Veganismus im Fokus\\
			$\rightarrow$ ``Ist das Produkt tatsächlich vegan?''
		\column{.40\textwidth}
			\centering
			% Quelle: siehe unten (500px-Version)
%			\includegraphics[scale=0.3]{pics/{500px-Veganismus_logo.svg}.png}
	\end{columns}
	
	\begin{block}{}
		wie ``Allergiewarner'', nur mit ethischem Hintergrund
	\end{block}
	
	\begin{alertblock}{}
		Wenn Gesetz geändert wird, kann dieser Ansatz für andere
		Ansätze verwendet werden
	\end{alertblock}
	
	\vskip0pt plus 1filll
	\par\hrulefill\par
	\tiny{Quelle:
	\url{http://de.wikipedia.org/wiki/Datei:Veganismus_logo.svg}}
\end{frame}

\begin{frame}{Gemeinfreiheit}
	\vskip0pt plus 1filll
		\begin{columns}
		\column{.60\textwidth}
			(fast) alle Daten gemeinfrei erhältlich\\
			$\rightarrow$ Open Data
		\column{.40\textwidth}
			\centering
			% Quelle: siehe unten
%			\includegraphics[scale=0.1]{pics/{zero.large}.png}
	\end{columns}
	\vskip0pt plus 1filll
	\begin{block}{CC0 1.0 Universell (CC0 1.0) - Public Domain Dedication}
		Sie dürfen das Werk/den Inhalt kopieren, verändern, verbreiten
		und aufführen, sogar zu kommerziellen Zwecken, ohne um weitere
		Erlaubnis bitten zu
		müssen.\,\footnote{\,\url{http://creativecommons.org/publicdomain/zero/1.0/deed.de}}
	\end{block}
	\vskip0pt plus 1filll
	\par\hrulefill\par
	\tiny{Quelle: \url{http://creativecommons.org/about/downloads}}
\end{frame}

\begin{frame}{Plattformunabhängigkeit}
	\vskip0pt plus 1filll
	Webanwendung

	\begin{center}
		% Quelle: siehe unten
%		\includegraphics[scale=0.08]{pics/browser-logos-banner.png}
	\end{center}

	Mobile App

	\begin{center}
		% Quelle: siehe unten (modifiziert)
%		\includegraphics[scale=0.3]{pics/Build-Diagram-3.png}
	\end{center}
	\vskip0pt plus 1filll
	\par\hrulefill\par
	\tiny{Quellen:
			\url{http://www.thekuroko.com/the-browser-is-where-its-at/},
	\url{http://phonegap.com/about/artwork/}}
\end{frame}

\section{Konzept}
\subsection*{Konzept}
\begin{frame}{Aufbau}
	\vskip0pt plus 1filll
		% Quellen: siehe unten. Zusaetzlich noch Bilder (Pfeil, Barcode)
		% und Texte von mir
%		\includegraphics[scale=0.15]{pics/konzept.png}
	\vskip0pt plus 1filll
	\par\hrulefill\par
	\tiny{Quellen:
	\url{http://www.minecraftwiki.net/wiki/File:Smartphone.svg},
	\url{https://www.google.com/intl/en/chrome/browser/features.html\#security},
	\url{https://commons.wikimedia.org/wiki/File:Database.svg}}
\end{frame}

\begin{frame}{Aufbau}
	\begin{itemize}
		\item Eintragung bzw. Veränderung von Daten nach Anmeldung mit
				OpenID möglich
		\item Jede Eintragung gibt Punkte\\
			$\rightarrow$ ``spielerisch'' die Datenbank füllen
		\item Daten (keine persönlichen) können per API/Dump
				heruntergeladen werden
	\end{itemize}
\end{frame}

\subsection*{Veganität}
\begin{frame}{Veganität (Berechnung)}
	\vskip0pt plus 1filll
	\centering
		% Quellen: Veganismus Logo (siehe oben) und
		% Rahmen/Beschriftungen von mir
%		\includegraphics[scale=0.2]{pics/veganitaet.png}
	\vskip0pt plus 1filll
\end{frame}

\begin{frame}{Veganität (Anzeige)}
	\vskip0pt plus 1filll
	\textbf{Noname Chips}\\
	Zutaten: Kartoffeln, Sonnenblumenöl, Salz, Aroma\\
	\begin{tabular}[b]{ c b{.8\textwidth} }
		% Quellen: Veganismus Logo (siehe oben) modifiziert und
		% exportiert
%		\includegraphics[scale=0.05]{pics/veganismus_logo_vegan.png}
		& Dieses Produkt ist von den Zutaten her vegan.\\
%		\includegraphics[scale=0.05]{pics/veganismus_logo_unsure.png}
		& Dieses Produkt hat noch keine Produktanfrage/Quelle.\\
%		\includegraphics[scale=0.05]{pics/veganismus_logo_unvegan.png}
		& Dieses Produkt wurde durch einen Kommentar als unvegan
		markiert.
	\end{tabular}
	\vskip0pt plus 1filll
\end{frame}

\section{Implementierung}
\subsection*{Webanwendung}
\begin{frame}{Webanwendung}
	\vskip0pt plus 1filll
		\begin{columns}
			\column{.3\textwidth}
				\centering
				% Quellen: siehe unten (Suche benutzen, jeweil
				% 500px-Version)
% 				\includegraphics[scale=0.1]{pics/{500px-HTML5_logo.svg}.png}\\
% 				\includegraphics[scale=0.1]{pics/{500px-CSS.svg}.png}\\
% 				\includegraphics[scale=0.1]{pics/{500px-Javascript_icon.svg}.png}
			\column{.3\textwidth}
				\centering
%				\includegraphics[scale=0.1]{pics/{500px-Ruby_on_Rails.svg}.png}
			\column{.3\textwidth}
				\centering
%				\includegraphics[scale=0.1]{pics/{500px-Postgresql_elephant.svg}.png}
		\end{columns}
	\vskip0pt plus 1filll
	\par\hrulefill\par
	\tiny{Quellen: \{commons,de,en\}.wiki\{pedia,media\}.org}
\end{frame}

%\subsection*{Phonegap}
\subsection*{Apps}
\begin{frame}{Apps (allgemein)}
	\vskip0pt plus 1filll
	\begin{center}
		% Quelle: siehe unten
%		\includegraphics[scale=0.2]{pics/Build-Diagram-2.png}
	\end{center}

	\vskip0pt plus 1filll
	\par\hrulefill\par
	\tiny{Quelle:
	\url{http://phonegap.com/about/artwork/}}
\end{frame}

\begin{frame}{Apps (YAVA)}
	\vskip0pt plus 1filll
	\begin{columns}
		\column{.6\textwidth}
			\begin{exampleblock}{Online}
				Nutzung der mobilen Website
			\end{exampleblock}
			\begin{block}{Offline}
				Schlanke Website mit SQLite-Datenbank\\
				Zuordnung: GTIN $\rightarrow$ Veganität
			\end{block}
		\column{.4\textwidth}
			\centering
			% Quelle: siehe oben (exportiert aus SVG)
%			\includegraphics[scale=0.2]{pics/smartphone.png}
	\end{columns}
	\vskip0pt plus 1filll
\end{frame}

\begin{frame}{Apps (IRL)}
	\vskip0pt plus 1filll
	{\large Innovative Retail Laboratory}
	\begin{columns}
		\column{.6\textwidth}
			\begin{block}{Offline}
				Schlanke Website mit Bild, Name und Veganität mit
				Testdaten in SQLite-Datenbank
			\end{block}
		\column{.4\textwidth}
			\centering
			% Quelle: das Logo wurde mir freundlicherweise zur
			% freien Verfuegung gestellt (d.h. gemeinfrei nutzbar)
			\includegraphics[scale=0.1]{pics/irl.png}
	\end{columns}

	\vskip0pt plus 1filll
	\par\hrulefill\par
	\tiny{Quellen: \url{http://www.innovative-retail.de/}}\\
	Spassova, L., Schöning, J., Kahl, G., \& Krüger, A. (2009).
	\textbf{Innovative retail laboratory.} In \textit{Roots for the Future of Ambient
	Intelligence. European Conference on Ambient Intelligence
	(AmI-2009), oA, Salzburg, Austria (November 2009).}
\end{frame}

\section{Ausblick}
\subsection*{futurework}
\begin{frame}{Ausblick}
	\begin{itemize}
		\item Orientierungshilfe
		\item Verwendung von Augmented Reality
		\item Dezentralisierung (z.\,B. mithilfe des P2P
				Frameworks\,\footnote{\,\url{http://p2pframework.com/}})
		\item API für Hersteller*innen zum Eintragen der Daten
		\item Unterstützung von Pharmaka bzw. Medikamenten
		\item Einbindung von weiteren Diensten
		\begin{itemize}
			\item Testberichte
			\item Preisvergleiche
			\item Öko- und Gesundheitsinformationen
		\end{itemize}
	\end{itemize}
\end{frame}

% don't include the following frames in the header (smoothbars)
% -> define another part
\part{End}

%% End %%

% Thanks
\begin{frame}{Vielen Dank ...}
\begin{center}
	... für eure Aufmerksamkeit!
\end{center}
\end{frame}

% Questions
\begin{frame}{Fragen}
\begin{center}
	% Quelle: selbst gemalt
	\includegraphics[scale=0.2]{pics/qm.png}
\end{center}
\end{frame}

\end{document}
