% http://www.kubieziel.de/computer/latex-tutorial.html
\documentclass[a4paper,11pt,bibtotoc,headsepline]{scrartcl}
\usepackage{outline} % include some stuff

\title{\namefull}
\subject{Outline\\
Bachelor-/Master-Seminar}
\author{Yannic Haupenthal (2522096)}
\date{\today{}}

% A written summary of no more than 3 pages in form of an extended
% abstract as preparation for the actual bachelor thesis (see sample)
% Reference: http://www.dfki.de/iui/bms/BA_MA_Outline.pdf

\begin{document}

\maketitle

\pagestyle{plain}

% Introduction
\section{Einleitung}

Als vegan lebender Mensch ist es nicht immer leicht im Supermarkt
vegane Produkte direkt zu erkennen, da keine gesetzliche
Deklarationspflicht besteht und somit die Produkte nicht
entsprechend gekennzeichnet werden müssen.

Daher müssen - wie auch z.\,B. bei Menschen mit Allergien - die Hersteller
angeschrieben bzw. gefragt werden, ob ein Produkt tatsächlich vegan
ist, da sich in diesem noch tierische Bestandteile befinden können,
die aber nicht direkt ersichtlich sind, da z.\,B. Aroma auf
pflanzlicher und tierischer Basis hergestellt werden kann, oder im
Herstellungsprozess unvegane ``Produkte'' benutzt wurden, wie es
z.\,B. bei der Klärung, d.\,h. der Herausfilterung von Trübstoffen in
der Wein- oder Saftherstellung der Fall ist.

\section{Grundlagen/Problemdefinition} % Basics/Problem definition

Um einen gleichen Kenntnisstand für alle zu ermöglichen, werden im
Folgenden kurz die grundlegenden Begriffe, die im weiteren Verlauf wichtig sind
geklärt: Vegetarismus, Veganismus und Veganität.

(Ovo-Lacto-)Vegetarismus bezeichnet eine Ernährungsweise, bei der keine
Lebewesen verzehrt werden, die ``Produkte'' dieser wie Milch (Lacto),
Eier (Ovo) und Honig aber schon.\\
Der Veganismus ist im Gegensatz dazu eine Lebensweise, die den Verzehr
aller tierischen ``Produkte'' ausschließt, und ebenso auch die Nutzung
von Tieren ablehnt.\\
Die Veganität bezeichnet, wie bzw. ob etwas vegan ist.

Die in der Einleitung erwähnte nicht vorhandene Deklarationspflicht
und das Problem, dass es bislang noch keine freie Produktdatenbank
gibt, führt dazu, dass alle bisher gestellten Produktanfragen an den
Hersteller bzgl. der Veganität eines Produktes in Foren bzw. Blogs
gesammelt werden, dies allerdings nicht zentral.

Die bisherige Vorgehensweise war dabei bei Menschen ohne Smartphone,
vor dem Kauf im Internet nach der Veganität zu suchen, bzw. bei Menschen
mit Smartphone während des Kaufs. Da die Suche nach der richtigen Produktanfrage
ziemlich zeitaufwändig sein kann, ist das nicht die präferierte
Herangehensweise.

Diese Arbeit soll diese Probleme lösen, indem eine freie
Produktdatenbank erstellt wird, die Produktanfragen zu den einzelnen
Produkten und Zutaten enthält und als Anwendung für verschiedene
Betriebssysteme verfügbar sein wird.\\
Produkte werden dabei anhand ihres Barcodes (GTIN) unterschieden.

\section{``Stand der Forschung''} % ``Research status''

Bisher wurden vier verwandte wissenschaftliche und kommerzielle
Arbeiten bzw. Dienste beschrieben und miteinander verglichen
\cite{arfhhm08, bre07, barcoo, did}.\\
Dabei wurde auf sechs Kriterien eingegangen, anhand derer sich diese
Arbeit von den anderen unterscheidet.

\section{Realisierung/Implementierung} % Realization/Implementation

Die fertige Arbeit soll aus einer Webanwendung und zwei
mobilen Anwendungen bestehen.

Die Webanwendung soll dabei auf dem Webframework Ruby-on-Rails
basieren, die plattformübergreifende Sprachen wie HTML5, CSS3 und
JavaScript verwendet.\\
Als Datenbank für alle Daten, die auch nachher - bis auf persönliche
wie Name und E-Mailadresse - frei
verfügbar sein sollen, soll PostgreSQL eingesetzt werden.

Die zwei mobilen Anwendungen (Apps) - \name\ und IRL genannt - werden
mit Phonegap erstellt, eine Software, die aus einer Codebasis und
Plugins native mobile Apps für alle gängigen mobilen Betriebssystem
erstellt.\\
\name\ soll dabei zwei Modi beinhalten, ein Online- und ein
Offline-Modus.\\
Der Online-Modus nutzt die mobile Version der Webanwendung, während
der Offline-Modus auf eine schlanke Website und eine SQLite-Datenbank
setzt, welche nur die Zuordnung GTIN zu Veganität beinhaltet.\\
Dies aus den einfachen Gründen, da nicht überall eine konstante
Internetverbindung vorliegt und die App mit der kompletten Datenbank
insgesamt viel zu groß wäre.\\
Die App namens IRL soll nur einen Offline-Modus besitzen. Diese App
wird dabei im Rahmen des Innovative Retail Laboratory (IRL)
- eine Kooperation von der Einzelhandelskette Globus mit dem
Deutschen Forschungszentrum für Künstliche Intelligenz (DFKI) -
eingesetzt, da die Daten direkt von Globus kommen und nicht
freigegeben werden dürfen. Die App wird dabei in dem IRL als
Demonstrator verwendet \cite{sskk09}.\\
Auch diese Variante basiert wie die Offline-Version von \name\ auf
einer schlanken Website mit SQLite-Datenbank, die Name, Bild, GTIN und
Veganität von Testdaten enthält.

Daten können nur nach einer Anmeldung erstellt oder verändert
werden.\\
Diese Anmeldung soll auf dem OpenID-Verfahren basieren und Dienste
wie Google, Facebook und Twitter unterstützen.\\
Aktionen wie Eintragung oder Änderung von Daten geben Punkte, damit
ein spielerischer Anreiz geschaffen wird, um die Datenbank schneller
mit Produktdaten zu füllen.

\section{Evaluation} % Evaluation (Auswertung/Feedback)

Eine Evaluation findet im Rahmen dieser Arbeit nicht statt.\\
Der Sinn der Bachelorarbeit lässt sich mit dem steigenden Wunsch der
Konsument*innen nach mehr Transparenz
nicht abstreiten, zudem es schon die genannten verwandten Arbeiten
gibt, die anstatt auf ethische Hintergründe z.\,B. auf Allergien setzen
und immer gefragter werden.\\
Auch die Frage nach dem Sinn einer offenen Produktdatenbank, die zudem noch für
andere Zwecke gebraucht werden kann, stellt sich in Hinsicht auf immer
mehr Open Data Projekte nicht.

\section{Änderungen seit Bachelorvortrag} % Changes since Bachelor-/Master-Talk

Nach dem Bachelorvortrag am 23.05.2013 im Rahmen des Bachelorseminars
wurden einige Fragen bzw. Anregungnen geäußert, die teilweise auch
Änderungsvorschläge waren.

Vorgeschlagen wurde u.\,a. ein sogenanntes ``Web-of-trust'', was dazu
dienen soll, die Arbeit der Datenpflege zu erleichtern, indem sie auf
mehrere Personen verteilt wird, die sich als vertrauenswürdig erwiesen
haben, da sie z.\,B. schon viele Punkte gesammelt haben.

Ebenfalls wurde die Frage gestellt, wie Synonyme oder Schreibfehler
behandelt werden. Angedacht war, dass Synonyme für eine Zutat oder ein
Produkt in der Datenbank gespeichert werden und Schreibfehler per
Levenshtein-Distanz korrigiert werden \cite{dam64}.

%\section{Literatur} % Literatur

\bibliographystyle{plain}
\renewcommand\refname{7 Literatur}
\bibliography{outline}

\end{document}
